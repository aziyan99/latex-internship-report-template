%-----------------------------------------------------------------------------%
\chapter{PENDAHULUAN}
\pagenumbering{arabic}
%-----------------------------------------------------------------------------%

%-----------------------------------------------------------------------------%
\section{Latar Belakang}
%-----------------------------------------------------------------------------%
Lorem ipsum dolor sit amet, consectetur adipiscing elit. Proin interdum in est sed imperdiet. Praesent nec mauris finibus, luctus risus ut, suscipit urna. Class aptent taciti sociosqu ad litora torquent per conubia nostra, per inceptos himenaeos. Nulla ut nulla ut dolor efficitur cursus. Pellentesque habitant morbi tristique senectus et netus et malesuada fames ac turpis egestas.

Quisque venenatis sapien sed vulputate tempus. Fusce sagittis enim eu dui vestibulum posuere. Orci varius natoque penatibus et magnis dis parturient montes, nascetur ridiculus mus. Suspendisse luctus metus ipsum, vel scelerisque sapien eleifend sit amet. Nulla nisi justo, tempus eu ante vitae, vestibulum sollicitudin orci.Fusce sagittis enim eu dui vestibulum posuere. Orci varius natoque penatibus et magnis dis parturient montes, nascetur ridiculus mus. Suspendisse luctus metus ipsum, vel scelerisque sapien eleifend sit amet. Nulla nisi justo, tempus eu ante vitae, vestibulum sollicitudin orci.

Orci varius natoque penatibus et magnis dis parturient montes, nascetur ridiculus mus. Suspendisse luctus metus ipsum, vel scelerisque sapien eleifend sit amet. Nulla nisi justo, tempus eu ante vitae, vestibulum sollicitudin orci.Class aptent taciti sociosqu ad litora torquent per conubia nostra, per inceptos himenaeos. Nulla ut nulla ut dolor efficitur cursus. Pellentesque habitant morbi tristique senectus et netus et malesuada fames ac turpis egestas.

%-----------------------------------------------------------------------------%
\section{Rumusan Masalah}
%-----------------------------------------------------------------------------%
Class aptent taciti sociosqu ad litora torquent per conubia nostra, per inceptos himenaeos. Nulla ut nulla ut dolor efficitur cursus. Pellentesque habitant morbi tristique senectus et netus et malesuada fames ac turpis egestas.

%-----------------------------------------------------------------------------%
\section{Tujuan Kerja Praktek}
%-----------------------------------------------------------------------------%
Nulla nisi justo, tempus eu ante vitae, vestibulum sollicitudin orci.Class aptent taciti sociosqu ad litora torquent per conubia nostra, per inceptos himenaeos. Nulla ut nulla ut dolor efficitur cursus. Pellentesque habitant morbi tristique senectus et netus et malesuada fames ac turpis egestas.

%-----------------------------------------------------------------------------%
\section{Manfaat Kerja Praktek}
%-----------------------------------------------------------------------------%
Nulla nisi justo, tempus eu ante vitae, vestibulum sollicitudin orci.Class aptent taciti sociosqu ad litora torquent per conubia nostra, per inceptos himenaeos. Nulla ut nulla ut dolor efficitur cursus. Pellentesque habitant morbi tristique senectus et netus et malesuada fames ac turpis egestas.
\begin{enumerate}
      \item Bagi Universitas \\
            Lorem ipsum dolor sit amet, consectetur adipiscing elit. Proin interdum in est sed imperdiet. Praesent nec mauris finibus, luctus risus ut, suscipit urna. Class aptent taciti sociosqu ad litora torquent per conubia nostra, per inceptos himenaeos.
      \item Bagi Mahasiswa \\
            Lorem ipsum dolor sit amet, consectetur adipiscing elit. Proin interdum in est sed imperdiet. Praesent nec mauris finibus, luctus risus ut, suscipit urna. Class aptent taciti sociosqu ad litora torquent per conubia nostra, per inceptos himenaeos.
      \item Bagi Instansi \\
            Lorem ipsum dolor sit amet, consectetur adipiscing elit. Proin interdum in est sed imperdiet. Praesent nec mauris finibus, luctus risus ut, suscipit urna. Class aptent taciti sociosqu ad litora torquent per conubia nostra, per inceptos himenaeos.
\end{enumerate}

%-----------------------------------------------------------------------------%
\section{Tempat dan Waktu Kerja Praktek}
%-----------------------------------------------------------------------------%

Lorem ipsum dolor sit amet, consectetur adipiscing elit. Proin interdum in est sed imperdiet. Praesent nec mauris finibus, luctus risus ut, suscipit urna. Class aptent taciti sociosqu ad litora torquent per conubia nostra, per inceptos himenaeos.

%-----------------------------------------------------------------------------%
\section{Sistematika Penyusunan}
%-----------------------------------------------------------------------------%

Dalam pembahasannya, penulis mengemukakan sistematika penulisan laporan yang terdiri dari lima bab sebagai berikut:

\vspace{0.5cm}
\noindent \textbf{BAB I PENDAHULUAN} \\
Berisikan tentang latar belakang masalah, perumusan masalah, tujuan kerja praktek, manfaat kerja praktek, dan sistematika pembahasan.

\vspace{0.5cm}
\noindent \textbf{BAB II PROFIL PERUSAHAAN} \\
Berisikan tentang gambaran umum dari PT. SUMATRA TIMURINDONESIA yang meliputi sejarah singkat perusahaan, budaya perusahaan, struktur organisasi perusahaan, visi dan misi serta nilai-nilai perusahaan.

\vspace{0.5cm}
\noindent \textbf{BAB III LANDASAN TEORI} \\
Berisikan tentang Teori-Teori yang berkaitan dengan topik yaitu:

\vspace{0.5cm}
\noindent \textbf{BAB IV DESKRIPSI SISTEM} \\
Bab ini menjelaskan tentang kegiatan pembuatan proyek yang dilakukan di tempat Kerja Praktek berupa :
\begin{enumerate}
      \item Deskripsi Sistem Berupa pemaparan tentang sebuah sistem yang digunakan untuk mempermudah mengelola data surat masuk dan keluar di PT. XYZ INDONESIA JAYA.
      \item Spesifikasi Sistem Berupa pemaparan tentang perangkat keras dan perangkat lunak yang digunakan selama pembuatan proyek.
      \item Perancangan Sistem Perancangan sistem terdiri dari: Perancangan basis data dan Data flow diagram
\end{enumerate}

\vspace{0.5cm}
\noindent \textbf{BAB V PEMBAHASAN} \\
Bab ini berisi tentang pembahasan dan analisis data secara deskriptif sistem yang sudah dipaparkan pada landasan teori.

\vspace{0.5cm}
\noindent \textbf{BAB VI PENUTUP} \\
Pada bab ini memberikan kesimpulan dan saran terhadap permasalahan yang timbul berdasarkan pengamatan penulis selama melakukan Kerja Praktek di PT. XYZ INDONESIA JAYA.
\begin{enumerate}
      \item \textbf{Kesimpulan} \\
            Kesimpulan ini berisi tentang rangkuman dari pelaksanaan pengerjaan proyek dan penulisan laporan akhir.
      \item \textbf{Saran} \\
            Terdiri dari saran-saran yang perlu diperhatikan selama pelaksanaan Kerja Praktek maupun saat pengerjaan proyek yang berupa masukan-masukan yang dapat membangun.
\end{enumerate}

\vspace{0.5cm}
\noindent \textbf{DAFTAR PUSTAKA} \\
Semua sumber kepustakaan yang ada dan digunakan dalam pelaksanaan proyek dan pembuatan laporan Kerja Praktek, yaitu baik berupa buku, lirik, maupun sumber-sumber lainnya yang terpercaya.

\vspace{0.5cm}
\noindent \textbf{LAMPIRAN} \\
Lampiran berisi tentang dokumentasi kegiatan selama kerja praktek.

