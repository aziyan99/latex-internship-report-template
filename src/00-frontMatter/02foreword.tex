%
% Foreword (Kata Pengantar)
%
% @author  Raja Azian
% @version 1.0.0
%

\chapter*{\centering KATA PENGANTAR}
\addcontentsline{toc}{chapter}{KATA PENGANTAR}

Prakata berisikan ucapan terima kasih penulis laporan kepada pihak-pihak yang telah memberikan kontribusi terhadap penulisan laporan, baik secara institusional maupun secara akademik.

Berikut adalah contoh penulisan rincian yang berisi ucapan terima kasih:
\begin{enumerate}
    \item Nama dosen pembimbing selaku dosen pembimbing lapangan
    \item Nama Mentor atau pembimbing di tempat kerja praktek
    \item Rekan-rekan ...
    \item Pihak lain yang tidak dapat disebutkan satu-persatu yang telah membantu baik secara langsung maupun tidak langsung, terima kasih atas segala dukungan yang diberikan kepada penulis.
\end{enumerate}

Dengan segala kerendahan hati, penulis menyadari bahwa laporan ini masih memiliki banyak kekurangan dan jauh dari kesempurnaan. Oleh karena itu, kritik dan saran yang membangun sangat dibutuhkan oleh penulis. Semoga laporan ini dapat bermanfaat bagi perkembangan ilmu dan teknologi serta bagi pihak-pihak yang membutuhkan.

\vspace{1cm}
\begin{flushright}
    Tanjungpinang, Juli \the\year{}\\
    \vspace{0.5cm}
    Penulis
\end{flushright}
\newpage
